% RESUMO--------------------------------------------------------------------------------

\begin{resumo}[RESUMO]
\begin{SingleSpacing}

% Não altere esta seção do texto--------------------------------------------------------
\imprimirautorcitacao. \imprimirtitulo. \imprimirdata. \pageref {LastPage} f. \imprimirprojeto\ – \imprimirprograma, \imprimirinstituicao. \imprimirlocal, \imprimirdata.\\
%---------------------------------------------------------------------------------------
\textit{Contexto:} A engenharia de \textit{software} é uma disciplina fundamental para cursos de tecnologia, com objetivo de prover organização, qualidade e produtividade no desenvolvimento de \textit{software}. A disciplina é ministrada predominantemente de maneira teórica, tornando as aulas tediosas e em muitos momentos desmotivantes. Para suprir este problema em cursos de Engenharia, este tema pode ser abordado de maneira diferente, tendo em vista o foco dessas áreas em circuitos e componentes eletrônicos. \textit{Objetivo:} Neste contexto, este trabalho visa a utilização da plataforma Arduino em conjunto com conceitos de gamificação, uma nova tendência para aumentar motivação, engajamento e desempenho, para apoiar o ensino de engenharia de \textit{software}. Em particular, será desenvolvida uma aplicação web utilizando conceitos de gamificação para apoiar o ensino de engenharia de \textit{software} e descrever os resultados obtidos em relação a sua utilização. \textit{Método:} A aplicação será composta por metas, regras, um sistema de \textit{feedback} e a utilização da mesma deverá ser de forma voluntária, caracterizando os quatro traços definidores de um jogo. Além disso, um questionário será aplicado para os alunos da disciplina, visando a obtenção dos resultados obtidos através da utilização da aplicação. 

\textbf{Palavras-chave}: Engenharia de Software. Arduino. Gamificação. Educação em Engenharia.

\end{SingleSpacing}
\end{resumo}

% OBSERVAÇÕES---------------------------------------------------------------------------
% Altere o texto inserindo o Resumo do seu trabalho.
% Escolha de 3 a 5 palavras ou termos que descrevam bem o seu trabalho 
