% FUNDAMENTACAO TEORICA------------------------------------------------------------------

\chapter{FUNDAMENTAÇÃO TEÓRICA}
\label{chap:fundamentacao-teorica}Este capítulo tem por finalidade a apresentação dos aspéctos teóricos e trabalhos relacionados, sendo constituído pelas seguintes seções: Seção 

\section{PLATAFORMA ARDUINO}
\label{sec:arduino} O arduino é uma plataforma de prototipagem eletrônica open-source criada em 2005, que baseia-se em hardware e software flexíveis e de fácil uso, tornando-se acessível para novatos e profissionais. O arduino é capaz de sentir o estado do ambiente através de sensores e interagir com o mesmo por meio de motores e outros atuadores, isto pode ser feito enviando um conjunto de instruções para o microcontrolador através da linguagem de programação Arduino e a IDE Arduino. Por ser uma plataforma open-source, possui uma vasta quantia de contribuições da comunidade mundial, composta por estudantes, programadores, artistas e profissionais, o que gera uma grande quantidade conhecimento acessível
