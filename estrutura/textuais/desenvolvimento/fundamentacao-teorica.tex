% FUNDAMENTACAO TEORICA------------------------------------------------------------------

\chapter{FUNDAMENTAÇÃO TEÓRICA}
\label{chap:fundamentacao-teorica}Este capítulo tem por finalidade a apresentação dos aspéctos teóricos e trabalhos relacionados, sendo constituído pelas seguintes seções: Seção 

\section{PLATAFORMA ARDUINO}
\label{sec:arduino} O arduino é uma plataforma de prototipagem eletrônica open-source criada em 2005, que baseia-se em hardware e software flexíveis e de fácil uso, tornando-se acessível para novatos e profissionais. O arduino é capaz de sentir o estado do ambiente através de sensores e interagir com o mesmo por meio de motores e outros atuadores, isto pode ser feito enviando um conjunto de instruções para o microcontrolador através da linguagem de programação Arduino e a IDE Arduino. Por ser uma plataforma open-source, possui uma vasta quantia de contribuições da comunidade mundial, composta por estudantes, programadores, artistas e profissionais, o que gera uma grande quantidade de conhecimento acessível, útil para usuários novatos e experientes \cite{arduino2018}.

O software Arduino é executado em sistemas operacionais Windows, Mac e Linux. Professores e alunos o utilizam para o desenvolvimento de instrumentos científicos de baixo custo, ou para introduzir a programação e robótica. \cite{arduino2018}.

\section{TESTE DE SOFTWARE E DEPURAÇÃO}
\label{sec:testedeploy} A computação evoluiu muito nas últimas décadas, sendo utilizada em diversas áreas da atividade humana, demandando qualidade e produtividade, e a engenharia de software acompanhou esse progresso, estabelecendo técnicas, critérios, métodos e ferramentas para o desenvolvimento de programas. A engenharia de software pode ser definida como uma disciplina que aplica os princípios da engenharia para a produção de softwares de alta qualidade e baixo custo.\cite{Pressman1997}. Apesar dos métodos e técnicas empregadas na produção de softwares, ainda podem ocorrer erros no produto, e com o intuito de minimizar a ocorrência destes, algumas atividades são introduzidas ao longo de todo o processo de desenvolvimento, sendo o teste de software a mais utilizada \cite{Maldonado1997}, constituindo-se em um dos elementos para oferecer evidencias da confiabilidade do software.

Os testes de softwares são compostos por quatro etapas: planejamento de testes, projeto de caso de testes, execução e avaliação dos resultados dos testes \cite{Maldonado2004}, que são executadas ao longo do desenvolvimento do software e efetivam-se em três diferentes fases de teste: de unidade, de teste e de sistema. Onde o primeiro, dedica-se na menor unidade do projeto, identificando erros de lógica e implementação em cada parte do programa. O teste de integração 
