% INTRODUÇÃO-------------------------------------------------------------------

\chapter{INTRODUÇÃO}
\label{chap:introducao}

O ensino de engenharia de \textit{software} deve proporcionar aos alunos experiências práticas profissionais da mesma, propiciando uma melhor compreensão de quais técnicas e práticas são melhores empregadas nas mais variadas situações \cite{Figueiredo2017}, se opondo da forma como é normalmente exposta em cursos de graduação, sendo comumente teórica \cite{Souza2010}. Quando apresentada de maneira prática, ocorre principalmente em projetos de sistemas de informação, área na qual não é o foco dos cursos de Bacharelado em Engenharia, que possuem uma abordagem maior em circuitos eletrônicos e componentes baseados no Arduino, o que torna o ensino de ES desmotivante e tedioso para estudantes desta área.

Além de experiências práticas, existem outras formas de incentivar o ensino, como a gamificação, um tema que vem despertando grande interesse nos últimos anos \cite{Kapp2012}, principalmente a sua aplicação na educação, que ganhou força devido ao seu grande sucesso obtido quando aplicado em outros contextos. Este tema, visa a utilização de elementos baseados em jogos e técnicas design de jogos, em contextos não relacionados a jogos, com intuito de engajar as pessoas, motivar a ação, aprimorar o aprendizado e a solucionar problemas em tarefas que poderiam ser consideradas desagradáveis \cite{Deterding2011}, \cite{Kapp2012}.

Mas afinal, o que é um jogo? Para \cite{McGonigal2011}, um jogo deve conter quatro traços definidores: uma meta, regras, um sistema de \textit{feedback} e participação voluntária. A meta dá um senso de propósito. As regras são essenciais para desencadear a criatividade e promover um pensamento estratégico. O \textit{feedback} propicia a motivação. A participação voluntária torna a experiência segura e prazerosa.

O desenvolvimento de sistemas confiáveis ainda é considerado um grande desafio para os programadores. A manutenção de \textit{software}, teste de \textit{software} e processo de \textit{software}, são os temas abordados neste trabalho. Ambos fazem parte da ES, estando presente no desenvolvimento de \textit{softwares}, e são fundamentais para a melhoria e adição de confiabilidade em sistemas.  

Este trabalho visa introduzir conceitos em engenharia de \textit{software} através da utilização de uma ?trocar(aplicação \textit{web})? composta por fases, divididas em três áreas da ES, onde o avanço em cada fase se dará após a submissão de uma atividade exigida, que será relacionada a um projeto utilizando a plataforma Arduino. Esta plataforma é muito utilizada em cursos de tecnologia em razão ao seu baixo custo, facilidade e popularidade na comunidade DIY (Do It Yourself). Prática que se baseia na construção de equipamentos eletrônicos, ou não, por conta própria, através de componentes disponíveis no mercado \cite{Kuznetsov2010}.



\section{MOTIVAÇÃO}
\label{sec:motivacao}


O crescente interesse pela gamificação gerou uma grande quantidade de estudos nesta área, devido ao seu grande sucesso em contextos não relacionados a educação. Devido a ES ser apresentada em disciplinas de maneira teórica, na maioria das vezes, pode fazer com que o aprendizado se torne tedioso e desmotivante para os alunos, acarretando em um baixo desempenho do mesmo nesta disciplina, e consequentemente, o baixo interesse por essa área no mercado de trabalho, área de extrema importância para o desenvolvimento de qualquer tipo de sistemas, por fomentar produtividade, organização e qualidade, aspectos que qualquer empresa de tecnologia almeja.

Alunos de Bacharelado em Engenharia, ligados à tecnologia da informação, como Engenharia de Computação, Engenharia Elétrica, Engenharia Eletrônica, Engenharia de Controle e Automação apresentam um grande interesse em manter o contato com circuitos e componentes eletrônicos, visto o foco destes cursos para essa área. Tendo em vista esse interesse, a plataforma Arduino é uma grande aliada para suprir a necessidade de contato com esses componentes eletrônicos, por ser barata e \textit{open-source}, possibilitando a sua customização, podendo fazer parte de projetos de certa complexidade por um baixo preço, além ainda, da motivação ocasionada para aulas que anteriormente poderiam ser entediantes para esses alunos, por se tratar de aulas teóricas.

Além disso, os jogos fazem parte da civilização há um bom tempo \cite{Borges2014} e estão presentes em sua maioria na forma digital nos dias atuais, principalmente na vida de crianças e jovens. Os quatros traços definidores de um jogo, segundo \cite{McGonigal2011}, os torna um ambiente perfeito para melhorar e provar nossas capacidades, levando nossas habilidades ao extremo, provendo um sentimento de realização, o oposto da depressão. Podendo também trabalhar a socialização, através da  cooperação (agir voluntariamente com um mesmo objetivo) e coo-criação (resolução de problemas em conjunto).
  


\section{OBJETIVOS}
\label{sec:objetivos}

 Dada a falta de interesse de alunos de Bacharelado em Engenharia para matérias de ES, este trabalho tem por finalidade utilizar a plataforma Arduino para introduzir conceitos desta matéria, através de um jogo educativo, e analisar de forma qualitativa se a gamificação e a utilização de componentes eletrônicos nas aulas contribuíram para uma melhor consolidação do conhecimento, motivação e engajamento dos alunos.
 
 Para tal análise, o jogo será utilizado por uma turma da disciplina de Engenharia de \textit{Software}, que irá utilizá-lo sem nenhum auxílio e posteriormente responderá um questionário, visando a obtenção de informações a respeito do emprego da ferramenta durante a aula.
 
 \section{ORGANIZAÇÃO TEXTUAL}
 \label{sec:orgTextual}
 
Este trabalho está organizado em 4 capítulos. Neste capítulo foram apresentados o contexto em qual o trabalho se insere, as motivações para a sua realização e os objetivos a serem alcançados. No capítulo \ref{chap:fundamentacao-teorica} relatam-se conceitos sobre a ES, enfatizando-se principalmente o teste de \textit{software}, manutenção de \textit{software} e processos de \textit{software}. No capítulo \ref{chap:proposta} são descritas as principais atividades a serem realizadas e por fim, o capítulo \ref{chap:cronograma} é apresenta o cronograma a ser cumprido.